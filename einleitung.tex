\section{Einleitung}
\label{sec:einleitung}

\subsection{Begründung der Problemstellung}

Durch Entwickungen wie \buzz{Internet of Things} und \buzz{Ubiquitous Computing} werden die in naher Zukunft die generierte Datenmenge als auch die Anzahl der verarbeitenden Instanzen in den nächsten Jahren wohl deutlich zunehmen.

\subsection{Ziele dieser Arbeit}

\textbf{Ziel dieser Arbeit ist es, die momentanen Entwicklungen der Datenbeständen hin zum semantischen Web und deren auswirkungen auf Wirtschaft und Gesellschaft mit den durch die Entdeckung des Öls im 20. Jahrhundert zu vergleichen.}

Hierzu werden zunächst im Kapitel~\myref{sec:grundlagen} die für diese Arbeit relevanten Begriffe und Konzepte definiert, bevor im Kapitel~\myref{sec:technologien} die momentan verfügbaren Technologien genannt und erklärt werden. 

Darauf aufbauend werden im Kapitel~\myref{sec:herausforderungen} die technischen Probleme identifiziert, und Lösungsansätze skizziert. Im Kapitel~\myref{sec:potentiale} werden schließlich einige mit dem IoT erwachsenden Möglichkeiten beschrieben. 

\todo{An die tatsächliche Arbeit anpassen}

\subsection{Abgrenzung}

\todo{An die tatsächliche Arbeit anpassen}
